% !TeX spellcheck = de_CH
\documentclass[11pt,a4paper,english,oneside]{book}

\usepackage{etex} %Because of many packages --> Extended TeX.
\usepackage[left=1in, right=1in]{geometry} %Helps to structure the paper layout.
\usepackage[Lenny]{fncychap} %Design of the thesis.
\usepackage[utf8]{inputenc} %Due to vowels.
\usepackage[T1]{fontenc}
\usepackage[ngerman]{babel} %Define the language style.

% german language for appendix header
\addto\captionsngerman{\let\appendixtocname\appendixname%
	\let\appendixpagename\appendixname}

% Setup glossary and acronyms
\usepackage[acronym,toc,nopostdot]{glossaries}
\loadglsentries{glossar-terms}
\makeglossaries

% Include external pdf
\usepackage{pdfpages}

\usepackage{dsfont} %Nice style for the indicator function.
\usepackage{fancyhdr} %To customize the headers and footers.
\usepackage{booktabs} %In case you need \cmidrule or \addlinespace in tables.
\usepackage[hang,bottom,stable,multiple]{footmisc} %Style of footnotes.
\usepackage{appendix} %For the \appendixpage command.
%Load some mathematical packages.
\usepackage{amsmath}
\usepackage{amsfonts}
\usepackage{amsmath}
\usepackage{amssymb}
\usepackage{mathtools}
\usepackage[
backend=biber,
style=numeric,
sortlocale=de_CH,
natbib=true,
doi=true,
eprint=false
]{biblatex}  % Bibliography
\addbibresource{References.bib}
\usepackage{etoolbox} %To remove the page number on \appendixpage.
\usepackage{amsthm} %For theorems, definitions etc.
\usepackage{thmtools} %For theorems, definitions etc.
\usepackage{setspace} %Use double spacing.
\usepackage{lipsum} %For the \lipsum command to generate a text.
\usepackage{datetime} %For the specification of the date.
%\usepackage{tocloft} %The ToC, LoF and LoT each appear not necessarily on a new page.
\usepackage{graphicx,listings,xcolor,textcomp} %For the graphics, listings etc.
\usepackage{mcode} %To implement a Matlab code.
\usepackage[margin=10pt, font=small, labelfont=bf, labelsep=endash]{caption} %Customize the captions.
\usepackage{chngcntr} %To use counterwithout.
\usepackage{epstopdf} %For inserting .eps files into the document.
\usepackage{hyperref} %Must be loaded at the end.
\usepackage{xparse} %Load for \NewDocumentCommand command.
\usepackage{cleveref} %For the command \cref, load after hyperref.
\usepackage{arydshln} %Due to the capability to draw horizontal/vertical dash-lines.
\usepackage{array,hhline} %To create tables and matrices.
\usepackage{rotating} %To rotate a table.
\usepackage{tabularx} %An extended version of tabular.


%Setup of the reference links.
\hypersetup{
	colorlinks=false,
	linkcolor=blue,
	citecolor=blue,
	filecolor=magenta,
	urlcolor=blue}




%Define some reasonable margins.
\setlength{\textwidth}{6.6in}
\setlength{\textheight}{8.8in}
\setlength{\topmargin}{-0.1in}
\setlength{\oddsidemargin}{0in}
\setlength{\parskip}{1mm}


\allowdisplaybreaks[1] %Page breaks of equations are allowed, but avoided if possible. 2-4 more relaxed.

%New command for the HSR logo.
\newcommand*{\plogo}{\includegraphics{logo_hsr.pdf}}

%New command for the differential d to have an ordinary d.
\makeatletter
\newcommand{\ud}{\mathrm{d}}
\makeatother

%Remove page number on \appendixpage. Use the package 'etoolbox'.
\makeatletter
\patchcmd{\@chap@pppage}{\thispagestyle{plain}}{\thispagestyle{empty}}{}{}
\makeatother

%Declare Definitions, Theorems etc.
%%%%%%%%%%%%%%%%%%%%%%%%%%%%%%%%%%%%%%%%%%%%%%%%%%%%%%%%%%%%%%%%%%%%%%%%%%%%%%%%%%%%%%%%%%%%%%%%%%%%%%%%%%%%%%%%%%%
\declaretheorem[style=definition,qed=$\blacktriangleleft$, numberwithin=chapter]{remark} %additional options; numberwithin=,..., see 'Thmtools' Users’ Guide
\declaretheorem[style=definition,qed=$\triangle$,numberwithin=chapter]{definition}
\newtheorem{ass}{Assumption}[chapter]
\newtheorem{prop}{Proposition}[chapter]
\newtheorem{lemma}{Lemma}[chapter]
\declaretheorem[style=definition,qed=$\perp$,numberwithin=chapter]{example}
\newtheorem{theorem}{Theorem}[chapter]
\newtheorem{coroll}{Corollary}[chapter]
%%%%%%%%%%%%%%%%%%%%%%%%%%%%%%%%%%%%%%%%%%%%%%%%%%%%%%%%%%%%%%%%%%%%%%%%%%%%%%%%%%%%%%%%%%%%%%%%%%%%%%%%%%%%%%%%%%%

%Readjust the numbering.
\counterwithout{footnote}{chapter}
\numberwithin{equation}{chapter}


%\setlength{\parindent}{0cm} %Uncomment this if you don't want to have indents.

%----------------------------------------------------------------------------------------
%	TITLE PAGE
%----------------------------------------------------------------------------------------
\newcommand*{\titleGP}{\begingroup %Create the command for including the title page in the document.
	\centering %Center all text.
	\vspace*{\baselineskip} %White space at the top of the page.
	\plogo\\[2\baselineskip] %University Logo.
	\rule{\textwidth}{1.6pt}\vspace*{-\baselineskip}\vspace*{2pt} %Thick horizontal line.
	\rule{\textwidth}{0.4pt}\\[\baselineskip] %Thin horizontal line.
	{\LARGE [ Graphs-Visualization-Service ]}\\[0.2\baselineskip] %Title.
	\rule{\textwidth}{0.4pt}\vspace*{-\baselineskip}\vspace{3.2pt} %Thin horizontal line.
	\rule{\textwidth}{1.6pt}\\[2\baselineskip] %Thick horizontal line.
	\scshape %Small caps.
	\large Studienarbeit \\[2\baselineskip]
	Abteilung Informatik \par
	Hochschule für Technik Rapperswil
	\vspace*{2\baselineskip}
	
	
	Autoren\\
	{\Large [ Michael Wieland ] \\ [5pt]}
	
	{\Large [ Murièle Trentini ] \\ [5pt]}
	
	\vspace*{2\baselineskip}
	Betreuer\\
	{\Large [ Thomas Letsch ] \\[5pt]
		\small Dozent für Informatik an der \\[5pt]Hochschule für Technik Rapperswil\par}
	\vspace*{2\baselineskip}
	
	Auditor\\
	{\Large [ Name ] \\[5pt]}
	\vspace*{2\baselineskip}
	
	\vfill
	{\scshape Zeitraum: 18.09.2017 - 22.12.2017} \\[0.3\baselineskip]
	\endgroup}


%Special header and footer style for the executive summary and Task Assignment section.
\fancypagestyle{firststyle}{%
	\fancyhf{}%
	\renewcommand{\headrulewidth}{0pt}
	\fancyfoot[C]{\thepage}}

%Customize headers and footers.
\pagestyle{fancy}
\fancyhead[R]{\thepage}
\fancyhead[L]{\rightmark}
\fancyfoot[L]{[Murièle Trentini | Michael Wieland ]}
\fancyfoot[C]{}
\fancyfoot[R]{[ GVS ]}

%Define the signature line with dots.
\NewDocumentCommand \dotbox {o O{.5\linewidth} m O{3ex} O{\linewidth}}
{
	\begin{minipage}{7cm}
		\makebox[7cm][l]{\,.\dotfill}
		\\
		\makebox[7cm][l]{\,#3}
	\end{minipage}
}

\begin{document}
	\thispagestyle{empty}
	\titleGP
	\newpage
	\doublespacing
	\setcounter{page}{1}
	\pagenumbering{Roman}
	\section*{Abstract}
	\thispagestyle{firststyle}
	
	
	\newpage
	
	\section*{Management Summary}
	\thispagestyle{firststyle}
	
	\subsection*{Ausgangslage}
	
	\subsection*{Vorgehen, Technologien}
	
	\subsection*{Ergebnisse}
	
	\subsection*{Ausblick}
	
	\newpage
	
	\section*{Technischer Bericht}  % ca 5 Seiten lang
	% Im Technischen Bericht sind formelle Elemente besonders wichtig (z.B. Bildunterschriften und das Referenzieren der Bilder im Text).
	\thispagestyle{firststyle}
	
	\subsection*{Ausgangslage \& Problembeschreibung}
	
	% (Beschreibung ob Fokus Lösungserstellung oder Machbarkeitsanalyse)
	
	\subsection*{Lösungskonzept}
	
	\subsection*{Umsetzung}
	
	\subsection*{Ergebnisdiskussion mit Ausblick}
	
	\newpage
	
	\section*{Danksagungen}
	\thispagestyle{firststyle}
	
	Wir danken folgenden Personen für Ihre Unterstützung während der Studienarbeit:
	
	\begin{itemize}
		\item Thomas Letsch für die Betreuung unserer Studienarbeit.
		\item Jessica Martin für die technische Unterstützung beim Logo Design.
	\end{itemize}
	
	\tableofcontents
	
	\newpage
	\pagenumbering{arabic}
	% \part{[ Part title ]}
	\chapter{[ Anforderungsanalyse ]}
	
	\begin{table}[h!]
		\centering
		\begin{tabularx}{\linewidth}{X X X X}
			\toprule 
			Datum & Version & Änderungen & Autor \\
			\midrule
			28.09.17 & 1.0 & Gerüst erstellt & wie \\
			\bottomrule 
		\end{tabularx} 
		\caption{Versionshistory Anfoderungsanalyse} 
	\end{table}
	
	\section{[ Ausgangslage ]}
	
	\section{[ Mehrwert ]}
	
	\section{[ Aufgabenstellung ]}
	
	\section{[ User Stories ]}
	
	\section{[ Use Cases ]}
	
	\subsection{Brief}
	
	\subsection{Fully Dressed}
	
	\section{[ Domainanalyse ]}
	
	
	
	%\part{ [ Realisierung ] }
	
	\chapter{ [ Realisierung ]}
	
		\begin{table}[h!]
		\centering
		\begin{tabularx}{\linewidth}{X X X X}
			\toprule 
			Datum & Version & Änderungen & Autor \\
			\midrule
			28.09.17 & 1.0 & Dokument erstellt & wie \\
			\bottomrule 
		\end{tabularx} 
		\caption{Versionshistory Realisierung} 
	\end{table}
	
	\section{ [ Architektur ] }
	
	\section{ [ UI Design ] }
	
	\subsection{Logo}
	Das Logo wurde von den Eigenschaften des Kraken \cite{kraken} inspiriert. Kraken sind bekannt dafür, dass sie viele Irrgarten-Probleme effizient lösen können. Dies ist eine Anspielung an die Algorithmen, die vom \gls{gvs} unterstützt werden. Ebenfalls wurden die Saugnäpfe des Kraken als Graph visualisiert und auf der Stirn ist ein binärer Baum zu erkennen. 
	
	\begin{figure}[h!]
		\centering
		\includegraphics[width=0.7\linewidth]{assets/images/logo}
		\caption[GVS Logo]{Graphs-Visualization-Service Logo}
		\label{fig:logo}
	\end{figure}
	
	
	
	
	
	\subsection{Konzept}
	
	\subsection{Icons}
	
	\subsection{Farben}
	
	\subsection{Wireframes}
	
	\section{ [ Testing ] }
	
	
	
	
	
	
	\chapter{ [ Projektmanagement ]}
	
	\begin{table}[h!]
		\centering
		\begin{tabularx}{\linewidth}{X X X X}
			\toprule 
			Datum & Version & Änderungen & Autor \\
			\midrule
			28.09.17 & 1.0 & Dokument erstellt & wie \\
			\bottomrule 
		\end{tabularx} 
		\caption{Versionshistory Projektmanagement} 
	\end{table}
	
	\section{ [ Zeitplanung ] }
	
	Das Projekt wird im Rahmen der Studienarbeit durchgeführt. Insgesamt stehen 14 Wochen zur Verfügung in welchen jedes Teammitglied 240 Stunden leisten muss. Somit entstehen ca. 17 Stunden Arbeitsaufwand pro Woche und Teammitglied.
		
	\section{ [ Projektverwaltung ] }
	
	Als Projektmanagement Software wird Jira \cite{jira} eingesetzt. In Jira werden alle Requirements als Issues erfasst und in den Product Backlog eingepflegt. Pro Iteration sollen jeweils so viele Issues eingeplant werden, wie unter Berücksichtigung von administrativen Aufgaben abgearbeitet werden können. Dabei spielt der Teamspeed eine grosse Rolle, welcher sich über die Projektdauer einpendeln soll. 
	
	\subsection{Phasen}
	Das Projekt ist grob in zwei Phasen aufgeteilt.
	
	\subsubsection{Analysephase}
	In der Analysephase wird das bestehende Produkt genau analysiert. Dabei soll festgestellt werden, welche Teile der Software verbessert werden müssen. Ebenfalls wird die Machbarkeit mit kleinen Prototypen validiert. 
	
	\subsubsection{Realisationsphase}
	In der Realisationsphase beginnt das eigentliche Software Projekt. Wie dabei vorgegangen wird ist in den folgenden Unterkapitel genauer beschreiben. 
	
	\subsection{Iterationsplanung}
	
	Die Iterationsplanung orientiert sich grob am \gls{rup} und ist unterteilt in eine Inception-, Elaboration-, Construction- sowie eine Transition-Phase. Jede Phase besteht aus einer oder mehreren Iterationen, die jeweils 2 Wochen dauern und am Freitag enden. Eine Iteration wird nach \gls{scrum} organisiert. Am Anfang des Sprints definiert das Projektteam die zu erledigen Issues und schätzt deren Aufwand. (Siehe \ref{sec:estimations})
	
	%TODO sync with artefact document
	%TODO allenfalls durch Artefact Overview (.xls) ersetzen
	\begin{table}[h!]
		\centering
		\begin{tabularx}{\linewidth}{l X }
			\toprule 
			Phase & Beschreibung \\
			\midrule
			Inception & Projektsetup (Jira, Workflow, IDE), Einarbeitung in die bestehenden Sourcen \\
			Elaboration 1 & Anforderungsspezifikation, Domainmodell \\
			Elaboration 2 & Architektur Modell, Guidelines für Quellcode und Tests, Wireframes \\
			Construction 1 & \\
			Construction 2 & \\
			Construction 3 &  \\
			Transition&   \\
			\bottomrule 
		\end{tabularx} 
		\caption{Iterationsplanung}
	\end{table}

	\subsection{Meilensteine}
	%TODO Releases und meilensteine hinzufügen
	
	\subsection{Schätzungen}
	\label{sec:estimations}
	Issues werden auf Basis von Story Points geschätzt. Dieses Vorgehen hat sich mit SCRUM etabliert. Die Nutzung von Story Points führt dazu, dass nicht die individuell unterschiedliche Bearbeitungszeit, sondern die Komplexität einer User Story geschätzt wird. Dies vereinfacht  und homogenisiert die Schätzungen und hilft den Teamspeed zu bestimmen.\cite{storypoints, storypoints2}
	
	\subsection{Zeitauswertung}
	Für die Zeitauswertung wird das Jira Plugin Tempo \cite{jiratempo} verwendet. Dieses bietet umfassende Auswertungsmöglichkeiten, sowie Exports nach MS Excel. Die Auswertungen dieser Arbeit befinden sich im Anhang \ref{zeitauswertung}.
	
	\subsection{Meetings}
	Über die gesamte Projektdauer findet jeweils am Mittwoch um 17:15 ein wöchentliches Standortmeeting statt. Die Beschlüsse aus den Meetings werden protokolliert und bis spätestens 24h später an alle Teilnehmer versendet. Allfälliges Feedback wird nachträglich eingepflegt und versioniert abgelegt.
	

	
	\section{ [ Artefakte ] }
	%TODO uebersicht einfügen
	
	\section{ [ Repositories ] }
	Die Artefakte werden in verschiedenen Repositories auf GitHub versioniert abgelegt. So gibt es für die Dokumentation, die Meetingprotokolle und den Sourcecode jeweils ein eigenes Repository.
	
	\subsection{Gitflow}
	Die Branch Struktur orientiert sich an Gitflow \cite{gitflow}. Pro Issue wird ein Feature Branch erstellt, der nach erfolgreichem Review in den Development Branch gemerged wird. Beim Erreichen eines Meilensteins wird der Entwicklungszweig in den Master Branch gemerged.
	
	\section{ [ Entwicklungsumgebung ] }
	Als \gls{ide} wird Eclipse mit folgenden Plugins verwendet
	\begin{itemize}
		\item FindBugs \cite{findbugs} zur statischen Code Analyse
		\item EclEmma \cite{eclemma} zur Überprüfung der Test Abdeckung
		\item Stan4J \cite{stan4j} zur Überprüfung von zyklischen Abhängigkeiten
		\item Checkstyle \cite{checkstyle} zur Überprüfung der Coding Richtlinien.
	\end{itemize}

	\section{ [ Frameworks ] }
	%TODO any additional libraries? 
	Als UI Framework wird gemäss Aufgabenstellung \gls{javafx} eingesetzt (siehe Anhang \ref{aufgabenstellung}). \gls{javafx} gilt seit 2014 als Standardlösung für grafische Java Anwendungen. \gls{javafx} ist eine komplette Neuentwicklung und der offizielle Nachfolger vom \gls{awt} \cite{awt} und Swing \cite{swing}. 

	
	\section{ [ Continuous Integration ] }
	%TODO Tavis
		
	\section{ [ Qualitätsmanagement ] }
	
	\subsection{TDD}
	Zur Sicherung der Software Qualität sind gute Tests unerlässlich. Für jedes Feature werden deshalb zuerst entsprechende Tests geschrieben. Dieses Vorgehen wird \gls{tdd} genannt.
	
	\subsection{Definition of Done}
	Source Code wird erst zum Review freigegeben, wenn folgende Kriterien erfüllt sind.
	\begin{itemize}
		\item Die \gls{ide} zeigt keine Warnungen und Fehler
		\item Es gibt keinen auskommentierten Code
		\item Es existieren sinnvolle Unit und Integrationstests für das Feature
		\item Alle Metriken geben grünes Licht
		\item Das Issue wurde in Jira \cite{jira} zum Review vermerkt
	\end{itemize}
		
	\subsection{Review}
	Nach Abschluss eines Issues wird dieses in Form eines Pull-Request an den Teampartner zum Review freigegeben. Durch das Vier-Augen-Prinzip kann die Qualität des Produktes hoch gehalten werden. Zusätzlich haben beide Teampartner stets den selben Wissensstand.
	
	\subsection{Metriken}
	%TODO Test Coverage, Code Style, Code Climate
		
	\section{ [ Risikomanagement ] }
	
	\subsection{Backups}
	Zur Minimierung von allfälligen Datenverlusten wird wie folgt vorgegangen:
	
	\begin{enumerate}
		\item Täglich automatisiertes Backup aller Projektdaten im JIRA \cite{jira} (Zeiterfassung, erstellte Issues, Jira Konfiguration)
		\item Die Projektdokumentation sowie der Programmcode wird in den vier Github Repositories der Github Organisation \cite{github} versioniert abgelegt. Für die Projektmitglieder gilt der Grundsatz ''Commit early and often''.
		\item Weitere Artefakte werden auf Dropbox \cite{dropbox} abgelegt und dort automatisch gesichert und versioniert.
	\end{enumerate}
	
	
	\newpage
	
	\appendix
	\noappendicestocpagenum
	\addappheadtotoc
	\appendixpage
	
	% Glossary
	\printglossary
	\glsaddall
	
	% List of references
	\printbibliography[heading=bibintoc]
	
	\listoffigures
	
	\listoftables
	
	\chapter{Eigenständigkeitserklärung}
	Der/Die Verfasser/in erklärt hiermit, dass er/sie die vorliegende Arbeit selbständig, ohne fremde Hilfe und ohne Benutzung anderer als die angegebenen Hilfsmittel angefertigt hat. Die aus fremden Quellen (einschliesslich elektronischer Quellen) direkt oder indirekt übernommenen Gedanken sind ausnahmslos als solche kenntlich gemacht. Durch Copyright geschützte Materialien wurden in dieser Arbeit nicht in unerlaubter Weise genutzt. Die Arbeit ist in gleicher oder ähnlicher Form oder auszugsweise im Rahmen einer anderen Prüfung noch nicht vorgelegt worden.\\[2cm]
	\dotbox{Ort, Datum} \hfill \dotbox{Unterschrift des/der Verfassers/in}
	\hfill \\[2cm]
	\dotbox{Ort, Datum} \hfill \dotbox{Unterschrift des/der Verfassers/in}
	
	\chapter{Vereinbarung}
	
	Mit dieser Vereinbarung werden die Rechte über die Verwendung und die Weiterentwicklung der Ergebnisse der Studienarbeit GVS von Muriele Trentini und Michael Wieland unter der Betreuung von Thomas Letsch geregelt.\\
	Die Urheberrechte stehen der Studentin / dem Student zu.\\
	Die Ergebnisse der Arbeit dürfen sowohl von der Studentin / dem Student wie von der HSR nach Abschluss der Arbeit verwendet und weiter entwickelt werden\\[2cm]
	\dotbox{Ort, Datum} \hfill \dotbox{Unterschrift des/der Verfassers/in}
	\hfill \\[2cm]
	\dotbox{Ort, Datum} \hfill \dotbox{Unterschrift des/der Verfassers/in}
	
	\chapter{Aufgabenstellung}
	\label{aufgabenstellung}
	Die folgenden Seiten enthalten die offizielle Aufgabenstellung dieser Studienarbeit.
	
    \includepdf[pages={1-}]{assets/Aufgabenstellung_GVS2.pdf}

	
	\chapter{Zeitauswertung}
	\label{zeitauswertung}
	
	\chapter{Testspezifikation}
	
	\chapter{Meeting Protokolle}
	
	
	
\end{document}

