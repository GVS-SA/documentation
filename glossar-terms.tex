% This document contains all glossary terms and acronyms

% Usage
% ---------------------
% \newglossaryentry{_label_}{name={_key_}, description={_value_}} creates only
% \newacronym{_label_}{_abbrv_}{_full_}

% \gls{_label_} 		-> Standard command
% \Gls{_label_}			-> Capitalize first letter
% \glspl{_label_}		-> Pluralize term
% \Glspl{_label_}		-> Capitalize and pluralize term

% Compile document with F9
% ---------------------

\newglossaryentry{gvs}
{
	name={GVS},
	description={Graphs-Visualization-Service: Name der Software die in der vorliegenden Studienarbeit entwickelt wird.},
	first={GVS (Graphs-Visualization-Service)}
}

\newglossaryentry{gvsui}
{
	name={GVS UI},
	description={Graphs-Visualization-Service UI: Komponente, die für die Visualisierung der Datenstrukturen verantwortlich ist. Enthält den Socket Server.},
	first={GVS UI}
}

\newglossaryentry{gvslib}
{
	name={GVS Lib},
	description={Graphs-Visualization-Service Lib: Komponente, die beim Client in das Software Projekt eingebunden werden muss. Enthält den Socket Client.},
	first={GVS Lib}
}


\newglossaryentry{hsr}
{
	name={HSR},
	description={Hochschule für Technik Rapperswil: Fachhochschule in Rapperswil an welcher die Studienarbeit durchgeführt wird.},
	first={HSR (Hochschule für Technik Rapperswil)}
}

\newglossaryentry{rup}
{
	name={RUP},
	description={Rational Unified Process: Vorgehensmodell in der Software Entwicklung, dass ursprünglich von IBM entwickelt wurde.},
	first={Rational Unified Process (RUP)}
}

\newglossaryentry{scrum}
{
	name={SCRUM},
	description={Verbreitetes Vorgehensmodell zur agilen Softwareentwicklung. Beinhaltet itertives Vorgehen mit stetigem Kundenfeedback.},
	first={SCRUM}
}

\newglossaryentry{javafx}
{
	name={JavaFX},
	description={Framework zur Erstellung von plattformübergreifenden grafischen Anwendungen in Java. Es ist eine komplette Neuentwicklung und seit 2014 der offizielle Nachfolger vom \gls{awt} und \gls{swing}},
	first={JavaFX}
}


\newglossaryentry{fxml}
{
	name={FXML},
	description={JavaFX XML: Dateiformat für JavaFX spezifische Views. Die beschreibende Syntax ähnelt jeder von XML.},
	first={FXML (JavaFX XML)}
}

\newglossaryentry{xml}
{
	name={XML},
	description={XML: Auszeichnungssprache zur Darstellung hierarchisch strukturierter Daten im Format einer Textdatei},
	first={XML (Extensible Markup Language)}
}

\newglossaryentry{ide}
{
	name={IDE},
	description={Integrated Development Environment: Tools zur effizienten Entwicklung von Software Produkten},
	first={IDE (Integrated Development Environment)}
}

\newglossaryentry{awt}
{
	name={AWT},
	description={Abstract Window Toolkit: In die Jahre gekommenes GUI Toolkit zum Erstellen von grafischen Java Applikationen.},
	first={AWT (Abstract Window Toolkit)}
}

\newglossaryentry{swing}
{
	name={Swing},
	description={GUI Toolkit zur Programmierung von grafischen Benutzeroberflächen. Swing baut auf dem älteren \gls{awt} auf.},
	first={Swing}
}

\newglossaryentry{tangle}
{
	name={Tangle},
	description={Dependencies von Packages aus tiefen Schichten auf Packages in höher gelegenen Schichten. So zeigt z.B. das Framework SonarQube einen ''Tangle Index'' an.},
	first={Tangle}
}


\newglossaryentry{jar}
{
	name={JAR},
	description={Java Archive: Container für ausführbare Java Files},
	first={JAR (Java Archive)}
}

\newglossaryentry{mvvm}
{
	name={MVVM},
	description={Model View ViewModel: Entwurfsmuster für die Darstellung von modernen UI-Plattformen mittels Databinding. MVVM ist eine Variante von \gls{mvc}.},
	first={MVVM (Model View ViewModel)}
}

\newglossaryentry{observer}
{
	name={Observer Pattern},
	description={Observer-Pattern: Das Pattern beschreibt eine Produzenten/Konsumenten Beziehung zwischen zwei Objekten. Der Konsument registriert sich beim Produzenten und wird automatisch notifiziert, wenn beim Produzenten neue Daten vorliegen.},
	first={Observer Pattern}
}

\newglossaryentry{mvc}
{
	name={MVC},
	description={Model View Controller: Verbreitetes Softwaremuster zur Trennung der Aufgabenbereiche von Software Komponenten die das UI steuern.},
	first={MVC (Model View Controller)}
}

\newglossaryentry{wpf}
{
	name={WPF},
	description={Windows Presentation Foundation: Klassenbibliothek zur Gestaltung von grafischen Benutzeroberflächen in .NET},
	first={WPF (Windows Presentation Foundation)}
}

\newglossaryentry{pojo}
{
	name={POJO},
	description={Plain Old Java Object: Triviales Java Objekt mit wenigen oder keinen externen Abhängikeiten},
	first={POJO (Plain Old Java Object)}
}

\newglossaryentry{dry}
{
	name={DRY},
	description={Don't Repeat Yourself: Software Prinzip zur Minimierung von Redundanzen im Code},
	first={DRY (Don't Repeat Yourself)}
}

\newglossaryentry{forcedirecteddrawing}
{
	name={Force Directed Drawing Algorithm},
	description={Algorithmus zur dynamischen Positionierung von Nodes mittels anziehenden und abstossenden Kräften. },
	first={Force Directed Drawing Algorithm}
}



