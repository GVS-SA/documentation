% This document contains all glossary terms and acronyms

% Usage
% ---------------------
% \newglossaryentry{_label_}{name={_key_}, description={_value_}} creates only
% \newacronym{_label_}{_abbrv_}{_full_}

% \gls{_label_} 		-> Standard command
% \Gls{_label_}			-> Capitalize first letter
% \glspl{_label_}		-> Pluralize term
% \Glspl{_label_}		-> Capitalize and pluralize term

% Compile document with F9
% ---------------------

\newglossaryentry{gvs}
{
	name={GVS},
	description={Graphs-Visualization-Service: Titel des vorliegenden Produktes},
	first={GVS (Graphs-Visualization-Service)}
}

\newglossaryentry{rup}
{
	name={RUP},
	description={Rational Unified Process: Vorgehensmodell in der Software Entwicklung, dass ursprünglich von IBM entwickelt wurde.},
	first={Rational Unified Process (RUP)}
}

\newglossaryentry{scrum}
{
	name={SCRUM},
	description={Vorgehensmodell zur agilen Softwareentwicklung},
	first={SCRUM}
}

\newglossaryentry{javafx}
{
	name={JavaFX},
	description={Framework zur Erstellung von plattformübergreifenden grafischen Anwendungen},
	first={JavaFX}
}

\newglossaryentry{fxml}
{
	name={FXML},
	description={JavaFX XML: Dateiformat für die JavaFX spezifischen Views},
	first={FXML (JavaFX XML)}
}

\newglossaryentry{xml}
{
	name={XML},
	description={XML: Auszeichnungssprache zur Darstellung hierarchisch strukturierter Daten im Format einer Textdatei},
	first={XML (Extensible Markup Language)}
}

\newglossaryentry{ide}
{
	name={IDE},
	description={Integrated Development Environment: Tools zur effizienten Entwicklung von Software Produkten},
	first={IDE (Integrated Development Environment)}
}

\newglossaryentry{awt}
{
	name={AWT},
	description={Abstract Window Toolkit: Veraltetes GUI Toolkit zum erstellen von grafischen Java Applikationen.},
	first={AWT (Abstract Window Toolkit)}
}

\newglossaryentry{jar}
{
	name={JAR},
	description={Java Archive: Container für ausführbare Java Files},
	first={JAR (Java Archive)}
}

\newglossaryentry{mvvm}
{
	name={MVVM},
	description={Model View ViewModel: Entwurfsmuster für die Darstellung von modernen UI-Plattformen mittels Databinding},
	first={MVVM (Model View ViewModel)}
}

\newglossaryentry{observer}
%TODO: allenfalls in eigenen Worten formulieren
{
	name={Observer Pattern},
	description={Observer-Pattern: 	Define a one-to-many dependency between objects so that when one object changes state, all its dependents are notified and updated automatically.},
	first={Observer Pattern}
}

\newglossaryentry{mvc}
{
	name={MVC},
	description={Model View Controller: },
	first={MVC (Model View Controller)}
}

\newglossaryentry{wpf}
{
	name={WPF},
	description={Windows Presentation Foundation: Grafik Framework von .NET},
	first={WPF (Windows Presentation Foundation)}
}

\newglossaryentry{pojo}
{
	name={POJO},
	description={Plain Old Java Object: Triviales Java Objekt mit wenigen oder keinen externen Abhängikeiten},
	first={POJO (Plain Old Java Object)}
}

\newglossaryentry{forcedirecteddrawing}
{
	name={Force Directed Drawing Algorithm},
	description={Force Directed Drawing Algorithm: Algorithmus zur dynamischen Positionierung von Nodes mittels anziehenden und abstossenden Kräften. },
	first={Force Directed Drawing Algorithm}
}



