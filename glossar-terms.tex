% This document contains all glossary terms and acronyms

% Usage
% ---------------------
% \newglossaryentry{_label_}{name={_key_}, description={_value_}} creates only
% \newacronym{_label_}{_abbrv_}{_full_}

% \gls{_label_} 		-> Standard command
% \Gls{_label_}			-> Capitalize first letter
% \glspl{_label_}		-> Pluralize term
% \Glspl{_label_}		-> Capitalize and pluralize term

% Compile document with F9
% ---------------------

\newglossaryentry{gvs}
{
	name={GVS},
	description={Graphs-Visualization-Service: Titel des vorliegenden Produktes},
	first={GVS (Graphs-Visualization-Service)}
}

\newglossaryentry{rup}
{
	name={RUP},
	description={Rational Unified Process: Vorgehensmodell in der Software Entwicklung, dass ursprünglich von IBM entwickelt wurde.},
	first={Rational Unified Process (RUP)}
}

\newglossaryentry{scrum}
{
	name={SCRUM},
	description={Vorgehensmodell zur agilen Softwareentwicklung},
	first={SCRUM}
}

\newglossaryentry{javafx}
{
	name={JavaFX},
	description={Framework zur Erstellung von plattformübergreifenden grafischen Anwendungen},
	first={JavaFX}
}

\newglossaryentry{ide}
{
	name={IDE},
	description={Integrated Development Environment: Tools zur effizienten Entwicklung von Software Produkten},
	first={IDE (Integrated Development Environment)}
}

\newglossaryentry{awt}
{
	name={AWT},
	description={Abstract Window Toolkit: Veraltetes GUI Toolkit zum erstellen von grafischen Java Applikationen.},
	first={AWT (Abstract Window Toolkit)}
}

\newglossaryentry{jar}
{
	name={JAR},
	description={Java Archive: Container für ausführbare Java Files},
	first={JAR (Java Archive )}
}


