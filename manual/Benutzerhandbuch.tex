% !TeX spellcheck = de_CH
\documentclass[11pt,a4paper,english,oneside]{book}

\usepackage{etex} %Because of many packages --> Extended TeX.
\usepackage[left=1in, right=1in]{geometry} %Helps to structure the paper layout.
\usepackage[Lenny]{fncychap} %Design of the thesis.
\usepackage[utf8]{inputenc} %Due to vowels.
\usepackage[T1]{fontenc}
\usepackage[ngerman]{babel} %Define the language style.

% german language for appendix header
\addto\captionsngerman{\let\appendixtocname\appendixname%
	\let\appendixpagename\appendixname}

% Include external pdf
\usepackage{pdfpages}

% set single pages to landscape
\usepackage{lscape}

\usepackage{dsfont} %Nice style for the indicator function.
\usepackage{fancyhdr} %To customize the headers and footers.
\usepackage{booktabs} %In case you need \cmidrule or \addlinespace in tables.
\usepackage[hang,bottom,stable,multiple]{footmisc} %Style of footnotes.
\usepackage{appendix} %For the \appendixpage command.
%Load some mathematical packages.
\usepackage{amsmath}
\usepackage{amsfonts}
\usepackage{amsmath}
\usepackage{amssymb}
\usepackage{mathtools}
\usepackage[
backend=biber,
style=numeric,
sortlocale=de_CH,
natbib=true,
doi=true,
eprint=false
]{biblatex}  % Bibliography
\addbibresource{References.bib}
\usepackage{etoolbox} %To remove the page number on \appendixpage.
\usepackage{amsthm} %For theorems, definitions etc.
\usepackage{thmtools} %For theorems, definitions etc.
\usepackage{setspace} %Use double spacing.
\usepackage{lipsum} %For the \lipsum command to generate a text.
\usepackage{datetime} %For the specification of the date.
%\usepackage{tocloft} %The ToC, LoF and LoT each appear not necessarily on a new page.
\usepackage{graphicx,listings,xcolor,textcomp} %For the graphics, listings etc.
\usepackage{adjustbox} %for step-by-step instructions
\usepackage{mcode} %To implement a Matlab code.
\usepackage[margin=10pt, font=small, labelfont=bf, labelsep=endash]{caption} %Customize the captions.
\usepackage{chngcntr} %To use counterwithout.
\usepackage{epstopdf} %For inserting .eps files into the document.
\usepackage{hyperref} %Must be loaded at the end.
\usepackage{xparse} %Load for \NewDocumentCommand command.
\usepackage{cleveref} %For the command \cref, load after hyperref.
\usepackage{arydshln} %Due to the capability to draw horizontal/vertical dash-lines.
\usepackage{array,hhline} %To create tables and matrices.
\usepackage{rotating} %To rotate a table.
\usepackage{dcolumn, tabularx, multirow} %An extended version of tabular.
\renewcommand{\arraystretch}{1.25}

\lstset{language=Java,keywordstyle={\bfseries \color{blue}}}%for inline code

%Setup of the reference links.
\hypersetup{
	colorlinks=true,
	linkcolor=cyan,
	anchorcolor=black,
	citecolor=cyan,
	filecolor=cyan,
	urlcolor=cyan,
	runcolor=cyan,
	menucolor=black
}


%Define some reasonable margins.
\setlength{\textwidth}{6.6in}
\setlength{\textheight}{8.8in}
\setlength{\topmargin}{-0.1in}
\setlength{\oddsidemargin}{0in}
\setlength{\parskip}{1mm}


\allowdisplaybreaks[1] %Page breaks of equations are allowed, but avoided if possible. 2-4 more relaxed.

%New command for the GVS logo.
\newcommand*{\plogo}{\includegraphics{logo.png}}

%New command for step-by-step instructions
\newenvironment{explanation}[2][]
{\begin{flushleft}
		\adjustbox{center=11cm,valign=c}{\includegraphics[width=10cm,#1]{#2}}%
		\begin{minipage}[c]{\dimexpr\textwidth-11cm\relax}}
		{\end{minipage}\end{flushleft}}

%New command for the differential d to have an ordinary d.
\makeatletter
\newcommand{\ud}{\mathrm{d}}
\makeatother

%Remove page number on \appendixpage. Use the package 'etoolbox'.
\makeatletter
\patchcmd{\@chap@pppage}{\thispagestyle{plain}}{\thispagestyle{empty}}{}{}
\makeatother



%Declare Definitions, Theorems etc.
%%%%%%%%%%%%%%%%%%%%%%%%%%%%%%%%%%%%%%%%%%%%%%%%%%%%%%%%%%%%%%%%%%%%%%%%%%%%%%%%%%%%%%%%%%%%%%%%%%%%%%%%%%%%%%%%%%%
\declaretheorem[style=definition,qed=$\blacktriangleleft$, numberwithin=chapter]{remark} %additional options; numberwithin=,..., see 'Thmtools' Users’ Guide
\declaretheorem[style=definition,qed=$\triangle$,numberwithin=chapter]{definition}
\newtheorem{ass}{Assumption}[chapter]
\newtheorem{prop}{Proposition}[chapter]
\newtheorem{lemma}{Lemma}[chapter]
\declaretheorem[style=definition,qed=$\perp$,numberwithin=chapter]{example}
\newtheorem{theorem}{Theorem}[chapter]
\newtheorem{coroll}{Corollary}[chapter]
%%%%%%%%%%%%%%%%%%%%%%%%%%%%%%%%%%%%%%%%%%%%%%%%%%%%%%%%%%%%%%%%%%%%%%%%%%%%%%%%%%%%%%%%%%%%%%%%%%%%%%%%%%%%%%%%%%%

%Readjust the numbering.
\counterwithout{footnote}{chapter}
\numberwithin{equation}{chapter}


%\setlength{\parindent}{0cm} %Uncomment this if you don't want to have indents.

%----------------------------------------------------------------------------------------
%	TITLE PAGE
%----------------------------------------------------------------------------------------
\newcommand*{\titleGP}{\begingroup %Create the command for including the title page in the document.
	\centering %Center all text.
	\vspace*{\baselineskip} %White space at the top of the page.
	\plogo\\[2\baselineskip] %University Logo.
	\rule{\textwidth}{1.6pt}\vspace*{-\baselineskip}\vspace*{2pt} %Thick horizontal line.
	\rule{\textwidth}{0.4pt}\\[\baselineskip] %Thin horizontal line.
	{\LARGE Graphs-Visualization-Service GVS 2.0 }\\[0.2\baselineskip] %Title.
	\rule{\textwidth}{0.4pt}\vspace*{-\baselineskip}\vspace{3.2pt} %Thin horizontal line.
	\rule{\textwidth}{1.6pt}\\[2\baselineskip] %Thick horizontal line.
	\scshape %Small caps.
	\Large Benutzerhandbuch
	\vspace*{2\baselineskip}
	
	
	Autoren\\
	{\large  Michael Wieland  \\ [5pt]}
	
	{\large Murièle Trentini \\ [5pt]}
	
	\vfill
	\endgroup}


%Special header and footer style for the executive summary and Task Assignment section.
\fancypagestyle{firststyle}{%
	\fancyhf{}%
	\renewcommand{\headrulewidth}{0pt}
	\fancyfoot[L]{GVS 2.0}
	\fancyfoot[R]{\thepage}}

% customize header and footer for chapter pages
\fancypagestyle{plain}{
	\fancyhf{}
	\fancyfoot[L]{GVS 2.0}
	\fancyfoot[R]{Benutzerhandbuch - \thepage}
	\renewcommand{\headrulewidth}{0pt}% Line at the header invisible
	\renewcommand{\footrulewidth}{0pt}% Line at the footer visible
}

%Customize headers and footers for normal pages
\pagestyle{fancy}
\fancyhead[R]{Murièle Trentini | Michael Wieland}
\fancyhead[L]{\rightmark}
\fancyfoot[L]{GVS 2.0}
\fancyfoot[C]{}
\fancyfoot[R]{Benutzerhandbuch - \thepage}

%Define the signature line with dots.
\NewDocumentCommand \dotbox {o O{.5\linewidth} m O{3ex} O{\linewidth}}
{
	\begin{minipage}{7cm}
		\makebox[7cm][l]{\,.\dotfill}
		\\
		\makebox[7cm][l]{\,#3}
	\end{minipage}
}

\begin{document}
	\thispagestyle{empty}
	\titleGP
	\newpage
	\setcounter{page}{1}
	\pagenumbering{Roman}
	{
		\hypersetup{linkcolor=black}
		\tableofcontents
	}
	
	%TODO Handbuch und Installationsanleitung. Kürzel in Fusszeile. Somit gibt es keine Verwirrung mit den Seitenzahlen
	
	\newpage
	\pagenumbering{arabic}
	
	\chapter{Anforderungen}
	Für die Inbetriebnahme von GVS 2.0 müssen folgende Anforderungen erfüllt sein.
	\begin{itemize}
		\item JDK 1.8 oder neuer
		\item Für die Weiterentwicklung des GVS 2.0 werden Grundkenntnisse von git vorausgesetzt
		\item Die folgende Installationsanleitung wurde für die Eclipse IDE geschrieben
	\end{itemize}

	\chapter{Installationsanleitung}
	\section{GVS UI ausführen}
	\subsection{Windows \& macOS}
	Das JAR File kann direkt per Doppelklick oder über die Konsole ausgeführt werden.
	
	\lstinline{java -jar gvs-ui.jar}
	
	\subsection{Linux}
	\paragraph{Oracle JDK}
	Für Linux Distributionen, die \textit{oraclejdk} verwenden, kann das JAR File ebenfalls direkt über die Konsole ausgeführt werden.
	
	\lstinline{java -jar gvs-ui.jar}
	
	\paragraph{Open JDK}
	Für Linux Distributionen, die \textit{openjdk} verwenden, wird zusätzlich die Installation von \textit{openjfx} benötigt.
	
	\begin{lstlisting}
	sudo dnf install java-1.8.0-openjdk-openjfx
	java -jar gvs-ui.jar
	\end{lstlisting}

	\clearpage
	
	\section{GVS Lib im Projekt integrieren}
	\subsection{Eclipse}
	\begin{enumerate}
		\item Rechtsklick auf das Projekt im \textit{Package Explorer}
		\item \lstinline{Build Path > Configure Build Path...}
		\item \lstinline{Tab ''Libraries'' > Add External JAR...}
		\item gvs-lib-java JAR File auswählen
		\item mit \lstinline{Apply and Close} bestätigen
	\end{enumerate}

	\subsection{Gradle}
	\begin{enumerate}
		\item gvs-lib-java JAR File in den Ordner \lstinline{src/main/resources/} kopieren
		\item In der Datei \textit{build.gradle}, die GVS Lib als Dependency hinzufügen
\begin{lstlisting}
dependencies {
	compile files('src/main/resources/libs/gvs-lib-java.jar')
}
\end{lstlisting}
	\end{enumerate}

	
	
	\chapter{Benutzeranleitung}
	\section{GVS Lib}
	Die folgenden Beispielfiles sowie weitere Testfiles sind im Repository \href{https://github.com/Graphs-Visualization-Service/gvs-tester}{gvs-tester} zu finden (siehe \ref{tbl:repos}).
	
	
	\subsection{Graphen}
	Zur Modellierung von Graph Elementen stehen folgende Klassen zur Verfügung.

	\begin{table}[h!]
		\centering
		\begin{tabularx}{\linewidth}{l l}
			\toprule 
			Interface & Nutzen \\
			\midrule
			GVSGraph & Ein Graph \\
			GVSDefaultVertex & Ein Vertex ohne Koordinaten. \\
			GVSRelativeVertex & Ein Vertex mit fixen Koordinaten (in Prozent) \\
			GVSDirectedEdge & Eine gerichtete Kante für Graphen \\
			GVSUndirectedEdge & Eine ungerichtete Kante für Graphen \\
			\bottomrule 
		\end{tabularx} 
		\caption{GVS Lib 2.0 Graph Interfaces} 
		\label{tbl:Interfaces}
	\end{table}
	
	Der Code Ausschnitt \ref{lst:helloWorldGraph} zeigt ein Beispiel, wie ein Graph erstellt und an das GVS UI gesendet werden kann.
	
\begin{lstlisting}[language=java, frame=single, caption={Hello World Graph}, label={lst:helloWorldGraph}]
import gvs.graph.GVSGraph;
import gvs.model.graph.TestDefaultVertex;
import gvs.model.graph.TestDirectedEdge;

public class HelloGraph {

	public static void main(String[] args) {
		GVSGraph graph = new GVSGraph("Hello World Graph");
		TestDefaultVertex v1 = new TestDefaultVertex("V1");
		TestDefaultVertex v2 = new TestDefaultVertex("V2");
		TestDirectedEdge e = new TestDirectedEdge(v1, v2, "V1 to V2");
		
		graph.add(v1);
		graph.add(v2);
		graph.add(e);
		
		graph.display();
		graph.disconnect();
	}
}
\end{lstlisting}	
	
	Durch jeden Aufruf von \lstinline{graph.display()} wird ein Snapshot des Graphen an das GVS UI gesendet. Ganz zum Schluss des Programms muss \lstinline{graph.disconnect()} aufgerufen werden.
	
	\subsubsection{Vertices und Edges}
	Für Vertices und Edges müssen ebenfalls konkrete Instanzen der Interfaces erstellt werden. Der Code Auszug \ref{lst:vertex-impl} zeigt ein Beispiel einer konkreten \textit{GVSDefaultVertex} Implementierung.
	
\begin{lstlisting}[language=java, frame=single, caption={DefaultVertex Implementierung}, label={lst:vertex-impl}]
import gvs.graph.GVSDefaultVertex;
import gvs.styles.GVSStyle;

public class TestDefaultVertex implements GVSDefaultVertex {
	private String label;
	private GVSStyle style;
	
	public TestDefaultVertex(String label) {
		this.label = label;
	}
	
	@Override
	public String getGVSVertexLabel() {
		return label;
	}
	
	@Override
	public GVSStyle getStyle() {
		return style;
	}
	
	public void setStyle(GVSStyle style) {
		this.style = style;
	}
}
\end{lstlisting}

	\clearpage
	
	\subsection{Trees}
	Zur Modellierung von Tree Elementen stehen folgende Klassen zur Verfügung.
	
	\begin{table}[h!]
		\centering
		\begin{tabularx}{\linewidth}{l X}
			\toprule 
			Interface & Nutzen \\
			\midrule
			GVSTreeWithRoot & Erstellt einen Tree anhand der Root Node und deren Kind Beziehungen \\
			GVSTreeWithCollection & Erstellt einen Tree anhand der Root Node und deren Kind Beziehungen. Die Collection muss manuell gefüllt werden.  \\
			GVSBinaryNode & Ein Vertex für Binary Trees \\
			\bottomrule 
		\end{tabularx} 
		\caption{GVS Lib 2.0 Tree Interfaces} 
		\label{tbl:tree-interfaces}
	\end{table}
	
	Der Code Ausschnitt \ref{lst:helloWorldTree} zeigt ein Beispiel, wie ein Tree erstellt und an das GVS UI gesendet werden kann.
	
\begin{lstlisting}[language=java, frame=single, caption={Hello World Tree}, label={lst:helloWorldTree}]
import gvs.business.tree.GVSTreeWithRoot;
import gvs.model.tree.TestBinaryNode;

public class HelloTree {
	
	public static void main(String[] args) {
		GVSTreeWithRoot tree = new GVSTreeWithRoot("Hello World Tree");
		
		TestBinaryNode root = new TestBinaryNode("root");
		TestBinaryNode left = new TestBinaryNode("left");
		TestBinaryNode right = new TestBinaryNode("right");
		
		root.setLeftChild(left);
		root.setRightChild(right);
		
		tree.setRoot(root);
		
		tree.display();
		tree.disconnect();
	}
}

\end{lstlisting}

	\subsubsection{Binary Nodes}
	Der Code Auszug \ref{lst:binarynode-impl} zeigt ein Beispiel einer konkreten \textit{GVSBinaryTreeNode} Implementierung.

\begin{lstlisting}[language=java, frame=single, caption={Binary Node Implementierung}, label={lst:binarynode-impl}]
public class TestBinaryNode implements GVSBinaryTreeNode {
	String label;
	GVSStyle style;
	GVSBinaryTreeNode leftChild;
	GVSBinaryTreeNode rightChild;
	
	public TestBinaryNode(String label, GVSStyle style) {
		this.label = label;
		this.style = style;
	}
	
	public TestBinaryNode(String label) {
		this(label, null);
	}
	
	@Override
	public String getNodeLabel() {
		return label;
	}
	
	@Override
	public GVSStyle getStyle() {
		return style;
	}
	
	@Override
	public GVSBinaryTreeNode getGVSLeftChild() {
		return leftChild;
	}
	
	@Override
	public GVSBinaryTreeNode getGVSRightChild() {
		return rightChild;
	}
	
	public void setRightChild(GVSBinaryTreeNode child) {
		this.rightChild = child;
	}
	
	public void setLeftChild(GVSBinaryTreeNode child) {
		this.leftChild = child;
	}
	
	public void setStyle(GVSStyle style) {
		this.style = style;
	}
	
	@Override
	public String toString() {
		return label;
	}
}
\end{lstlisting}

	\clearpage
	
	\subsection{Styles ändern}
	Um komplexe Algorithmen farblich zu visualisieren, bietet GVS 2.0 eine Style Klasse an. Wie Styles von Edges und Vertices verändert werden können, zeigt Code Ausschnitt \ref{lst:styles}
	
	\begin{lstlisting}[language=java, frame=single, caption={Styles verändern}, label={lst:styles}]
	// lineColor, lineStyle, lineThickness
	GVSStyle edgeStyle = new GVSStyle(GVSColor.RED, GVSLineStyle.DASHED,
	GVSLineThickness.BOLD);
	
	// lineColor, lineStyle, lineThickness, fillColor
	GVSStyle vertexStyle = new GVSStyle(GVSColor.BLUE, GVSLineStyle.DOTTED,
	GVSLineThickness.SLIGHT, GVSColor.GREEN);
	
	v1.setStyle(vertexStyle);
	e.setStyle(edgeStyle);
	\end{lstlisting}
	
	\subsection{Icons verwenden}
	GVS 2.0 nutzt zur Darstellung von Vertex Icons die Schriftart FontAwesome. Code Abschnitt \ref{lst:icons} zeigt, wie die Icons für Vertices verwendet werden können.
	
	\begin{lstlisting}[language=java, frame=single, caption={Icons benutzen}, label={lst:icons}]
	// lineColor, lineStyle, lineThickness, fillColor, icon
	GVSStyle iconStyle = new GVSStyle(GVSColor.STANDARD, GVSLineStyle.THROUGH,
	GVSLineThickness.STANDARD, GVSColor.STANDARD, GVSIcon.BICYCLE);
	v1.setStyle(iconStyle);
	graph.display();
	\end{lstlisting}	
	
	\clearpage
	
	\section{GVS UI}
		\begin{figure}[h!]
		\centering
		\includegraphics[width=0.7\linewidth]{assets/images/gvs-ui-graph}
		\caption{User Interface GVS 2.0}
		\label{fig:gvs-ui-graph}
	\end{figure}

	\subsection{User Interface}
	Alle wichtigen Funktionen von GVS 2.0 sind direkt über die Toolbar zugreifbar und alle Buttons verfügen über Tooltips. 
	
	\subsubsection{1. Toolbar}
	Über die Toolbar, die am oberen Fensterrand dargestellt wird, sind die grundlegenden Aktionen schnell und einfach zugänglich. So kann ein Benutzer eine gespeicherte Session vom Dateisystem laden, die aktuelle Session speichern oder löschen und zwischen den aktuellen Sessions aus der Dropdown-Box wechseln. 
	
	\subsubsection{2. Step Buttons und Fortschrittsanzeige}
	Die Step Buttons bilden zusammen mit der Replay Funktionalität ein wichtiges Werkzeug, um die Funktionsweise eines Algorithmus schrittweise zu analysieren und zu verstehen. Über die Buttons können die einzelnen Momentaufnahmen einer Session durchgeschaut werden.
	
	\subsubsection{3. Snapshot Description}
	Zur aktuellen Momentaufnahme können Notizen erfasst werden. Beim Speichern der Session werden diese ebenfalls gespeichert und werden beim Laden der Session wieder angezeigt.
	
	\subsubsection{4. Replay}
	Die Replay Funktionalität automatisiert die Aktionen der Step Buttons in einer bestimmten Geschwindigkeit. Standardmässig wird jede Sekunde zur nächsten Momentaufnahme gewechselt. Das Replay kann nach belieben gestoppt, gestartet und abgebrochen werden.
	
	\subsubsection{5. Autolayout}
	%TODO update description to new functionality (force layout)
	Das Autolayout steht nur für Graphen zur Verfügung. Alle Vertices die noch nicht mit der Maus positioniert wurden, werden vom Graph Layouter so positioniert, dass es möglichst wenig Überschneidungen der Edges gibt. Standardmässig verwendet der Graph Layouter immer zufällige Startkoordinaten. Dies kann über den ToggleButton deaktiviert werden, damit immer das gleiche Layout resultiert.
	
	\subsection{Drag Support}
	Bei Graphen können alle Vertices beliebig mit der Maus positioniert werden. Sobald alle Vertices manuell positioniert wurden, ist die Autolayout Funktion nicht mehr verfügbar. Trees können nicht manuell positioniert werden. Eine entsprechende Meldung wird angezeigt.
	
	\clearpage
	
	\section{Konfiguratiosfiles verändern}
	Um Änderungen an Konfigurationsfiles durchzuführen die im JAR File enthalten sind, muss folgende Anleitung befolgt werden:
	
	\begin{enumerate}
		\item JAR mit einem beliebigen File Archiver entpacken
		\item In den entstandenen Ordner wechseln
		\begin{enumerate}
			\item \lstinline{cd gvs-ui}
		\end{enumerate}
		\item gewünschte Files verändern
		\item Files wieder packetieren
		\begin{enumerate}
			\item \lstinline{jar cfm ../gvs-ui.jar META-INF/MANIFEST.MF gvs/GVSApplication.class *}
		\end{enumerate}
	\end{enumerate}

	Folgende Einstellungen können vorgenommen werden:
	
		\begin{table}[h!]
		\centering
		\begin{tabularx}{\linewidth}{l X}
			\toprule 
			File & Einstellung \\
			\midrule
			config.xml & Festlegen des Startports. Default: \lstinline{3000} \\
			gvs/ui/view/app/AppView.css & Farben, Dicke von Edges und Vertex-Rändern\\
			gvs/ui/view/session/SessionView.css & Schriftgrösse für Vertices und Edges\\
			logback.xml & Festlegen des Log-Levels (siehe \ref{ssec:logger})\\
			\bottomrule 
		\end{tabularx} 
		\caption{GVS Lib 2.0 Interfaces} 
		\label{tbl:preferences}
	\end{table}
	
	\subsection{Änderungen des Log-Level} \label{ssec:logger}
	\subsubsection{Im XML File}
	Der applikations-übergreifende Log Level kann wie folgt verändert werden:
	
	\begin{lstlisting}[language=xml, frame=single, caption={Root Log Level verändern}, label={lst:root-log-level}]
	<root level="DEBUG"> <!-- Log Level  -->
		<appender-ref ref="FILE" /> <!--  Log Output-->
		<appender-ref ref="STDOUT" /> <!--  Log Output-->
	</root>
	\end{lstlisting}
	
	Mögliche Level sind:
	\begin{itemize}
		\item OFF
		\item TRACE
		\item DEBUG
		\item INFO
		\item WARN
		\item ERROR
	\end{itemize}

	Wie das Log-Level package-spezifisch festgelegt werden kann, zeigt der Code Ausschnitt \ref{lst:log-level}.
	
	\begin{lstlisting}[language=xml, frame=single, caption={Root Log Level verändern}, label={lst:log-level}]
	<logger name="gvs.access" level="DEBUG">
		<appender-ref ref="FILE" />
	</logger>
	\end{lstlisting}	

	\subsubsection{Zur Laufzeit}
	Über die JConsole kann das Log Level auch zur Laufzeit verändert werden. Die JConsole kann einfach über die Konsole geöffnet werden: 
	
	\lstinline|$> jconsole|

	
	\begin{explanation}{assets/images/jconsole-1}
		gvs-ui aus Liste auswählen und mit \lstinline{Connect} bestätigen
	\end{explanation}
	\begin{explanation}{assets/images/jconsole-warning}
		Warnung über unsichere Verbindung bestätigen
	\end{explanation}
	\begin{explanation}{assets/images/jconsole-overview}
		Ansicht \textit{Overview} erscheint
	\end{explanation}
	\begin{explanation}{assets/images/jconsole-mbeans}
		Zu Ansicht MBeans wechseln
	\end{explanation}
	\begin{explanation}{assets/images/jconsole-setlevel}
		Links im Verzeichnis-Baum zu \lstinline{ch.qos.logback.classic >} \lstinline{GVS-Loggers >} \lstinline{Operations >} \lstinline{setLoggerLevel} wechseln
	\end{explanation}
	\begin{explanation}{assets/images/jconsole-parameters}
		Im oberen Fenster-Bereich unter \textit{Operation invocation} gewünschte Parameter angeben
		\begin{enumerate}
			\item Parameter: package (e.g. \lstinline{gvs.ui})
			\item Parameter: Log Level (e.g. \lstinline{WARN})
		\end{enumerate}
		Mit \lstinline{setLoggerLevel} bestätigen
	\end{explanation}

	\chapter{Weiterentwicklung}
	Sämtlicher Programmcode von GVS 2.0 ist frei auf Github verfügbar. Um Verbesserungen durchzuführen, beschreibt dieser Abschnitt die notwendigen Schritte um sich die Entwicklungsumgebung aufzusetzen.
	
	\section{Repositories}
	GVS 2.0 umfasst 4 Repositories, welche in Tabelle \ref{tbl:repos} aufgeführt sind.
	
	\begin{table}[h!]
		\centering
		\begin{tabularx}{\linewidth}{l l X}
			\toprule 
			Repository & Nutzen & URL \\
			\midrule
			gvs-ui & UI und Server & \url{https://github.com/Graphs-Visualization-Service/gvs-ui}  \\
			gvs-lib-java & Library für Java Programme & \url{https://github.com/Graphs-Visualization-Service/gvs-lib-java} \\
			gvs-lib-csharp & Library für C\# Programme & \url{https://github.com/Graphs-Visualization-Service/gvs-lib-csharp} \\
			gvs-tester & Ausführbare End-to-End Testfiles in Java & \url{https://github.com/Graphs-Visualization-Service/gvs-tester} \\
			\bottomrule 
		\end{tabularx} 
		\caption{GVS 2.0 Repositories} 
		\label{tbl:repos}
	\end{table}

\clearpage
	
	\section{Schritt für Schritt}
	Am Beispiel des gvs-ui Repositories folgt eine Schritt für Schritt Anleitung vom git clone bis zum fertigen Build. Diese Anleitung kann analog auch für die anderen Java Repositories benutzt werden.
	
	\begin{enumerate}
		\item \lstinline{git clone https://github.com/Graphs-Visualization-Service/gvs-ui.git}
		\item Importieren des Projekts in Eclipse
		\begin{enumerate}
			\item \lstinline{File > Import > Existing Gradle Project}
			\item als \textit{Project root directory} das geklonte Repository auswählen und mit \lstinline{Finish} bestätigen.
		\end{enumerate}
		\item Nötige Dependencies installieren
		\begin{enumerate}
			\item Rechtsklick auf den Projekt-Ordner im \textit{Package Explorer}
			\item \lstinline{Gradle > Refresh Gradle Project}
		\end{enumerate}
		\item Builden des Projekts mit Gradle
		\begin{enumerate}
			\item \lstinline{Window > Show View > Other}
			\item Im Suchfeld nach \textit{gradle} suchen
			\item \lstinline{Gradle Task} auswählen, mit \lstinline{Open} bestätigen
			\item In der neuen View sind alle Gradle Projekte aufgelistet
			\item \lstinline{gvs-ui > build > build} doppelklicken
			\item Der Build Task compiliert die Source-Files, lässt Tests laufen und erstellt ein ausführbares JAR im Projekt-Ordner \lstinline{ path/to/project/build/libs}
		\end{enumerate}
		\item Ausführen von GVS 2.0 UI
		\begin{enumerate}
			\item Entweder erstelltes JAR ausführen
			\item oder in Eclipse: Gradle Task \lstinline{application > run} ausführen
		\end{enumerate}
	\end{enumerate}
		
\end{document}

